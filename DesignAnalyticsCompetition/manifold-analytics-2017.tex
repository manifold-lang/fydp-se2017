\documentclass[12pt]{article}

\usepackage{fullpage,url,amssymb}
\usepackage{graphicx}
\usepackage[pdftex]{hyperref}
\usepackage{titlesec}

\usepackage{parskip}

\titleclass{\section}{top}
\pagestyle{empty}

\title{Manifold: a Language and Toolchain for Microfluidic Circuit Design}
\date{\vspace{-10ex}}

\begin{document}

\maketitle

{\bf Program:} Software Engineering

{\bf Members:} Nik Klassen ({\tt ndklasse@uwaterloo.ca}), Michael Lyons ({\tt mjlyons@uwaterloo.ca}), Michael Prysiazny ({\tt msprysia@uwaterloo.ca}), Paul Roth ({\tt phroth@uwaterloo.ca}), Peter Socha ({\tt psocha@uwaterloo.ca})

{\bf Faculty Advisor:} Derek Rayside ({\tt drayside@uwaterloo.ca})

\rule{\textwidth}{1pt}

% Instruction document: https://d1b10bmlvqabco.cloudfront.net/attach/ixk639stlzj1zq/h1rz9j1yc6n1uy/ixkfe4bda36q/2017_IDEAs_Design_Analytics_Competition.pdf

% Concise project overview goes here
% - Motivation and significance
% - Purpose of analysis

Manifold is an open-source high-level language for systems engineering.
Our capstone team has been extending a prototype language initially developed by an earlier capstone team from the SE 2015 cohort.
We have focused mainly on expanding Manifold's support for describing microfluidic circuits.

The field of microfluidic circuit engineering currently suffers from a lack of accessible software tools for design and analysis.
Engineers currently need to develop systems of equations that represent their circuits by hand.
They must then manually solve these equations to determine the system's viability.
With Manifold, engineers can describe their microfluidic systems in a functional programming style.
Manifold converts this system code into inputs for the dReal satisfiability solver and the MapleSim simulator, allowing engineers to test designs without needing to physically construct the circuits.

The goal of Manifold is to allow microfluidics engineers to easily find faults in their designs that would otherwise be uncovered only after a circuit's physical construction.
With the aid of an automated analysis tool, engineers will be able to iterate more quickly and more efficiently.

% TODO
% - Analysis approach, including analysis type, tools (software) used, and analysis verification
% - Summary of results to date
% - Conclusions and impact

\end{document}
